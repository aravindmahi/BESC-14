\subsection{Election Prediction}
\begin{table*}[ht]
	\centering
	\begin{tabular}{| l |l| r | r |}
		\hline
		Election Type & Winner& Seed Words & PSL Expansion\\
		\hline
		2012 Mexico & Enrique Peña Nieto&  &  \\
		2012 Venezuela & Hugo Ch\'{a}vez& &\\
		2013 Venezuela & Nicolás Maduro & & \\
		2013 Chile first round & Michelle Bachelet & True& \\
		2013 Chile second round & Michelle Bachelet  & True & \\
		2013 Ecuador & Rafael Correa & True& \\
		2013 Honduras & Juan Orlando Hernández  & False& \\
		2013 Paraguay & Horacio Cartes & True& \\
		2014 Colombia first round & Óscar Iván Zuluaga & False& \\
		2014 Colombia second round & Juan Manuel Santos & False& \\
		2014 Panama & Juan Carlos Varela& False& \\\hline
		Average Accuracy &    & \%& \%\\
		\hline
	\end{tabular}
	\caption{Track Record of Prediction Algorithms(Need to complete experiments on the missing countries)}
	\label{table:trackRecord}
\end{table*}

We adapt one basic forecasting algorithm from current literature to evaluate the
performance of our expanded vocabularies. The model uses a regression fit to map from tweet features to opinion polls and thus forecast elections. We dub this model as the \emph{``regression model"} and this approach is adapted from \cite{bermingham2011using} and \cite{o2010tweets}.

{\bf Regression Model:}
In this model, in addition to tweets, we leverage any opinion polls available for the elections 
to make our predictions.
Like the earlier model we track the tweets that mention words/phrases
from the vocabulary defined for each candidate.
We then conduct a linear regression fit that uses the opinion polls as dependent variable and features generated from 
these tweets as independent variable.
We reason that by regressing from the Twitter features to the opinion polls the bias due to Twitter being a non-representative sample
can be mitigated.
We use a total of six features: Klout scores, number of unique users, total number of mentions, sentiment, and incumbency.
The Klout scores, unique users, and mentions are further categorized into positive and negative mentions.
We normalize each of these features across all candidates to obtain the relative share of the volume. 
When there is more than one polling house publishing an opinion poll (for the same date) we take the average of the polls. 

\noindent
{\bf Performance:}
The Regression Model was tested exhaustively on a total of 11 presidential elections from Latin America during 2012 and 2014. we run the prediction model in two different setting: only using the seeding words and using PSL query expansion.The tweets were purchased from DataSift, an infoveilence service that resells Twitter data. Then these tweets were geo-coded using a geo-location algorithm we developed to obtain tweets from the country of interest.
Only tweets from the locations pertaining to elections were used to make the predictions.
For example, for the Colombia 2014 presidential elections only tweets from Colombia were used.
Once the tweets were filtered by location the time series of Klout and sentiment scores were calculated by tracking the tweets for the mentions of candidate.

Table~\ref{table:trackRecord} shows the overall performance of the regression model under two settings. 
\TODO{Wei}{Write something about the model according to the result}.
