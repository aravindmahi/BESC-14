\subsection{Election Prediction}
\begin{comment}

\begin{table*}[ht]
	\centering
	\begin{tabular}{| l |l| r | r |}
		\hline
		Election Type & Winner& Seed Words & PSL Expansion\\
		\hline
		2012 Mexico & Enrique Peña Nieto&  & True \\
		2012 Venezuela & Hugo Ch\'{a}vez& & True\\
		2013 Venezuela & Nicolás Maduro & & True\\
		2013 Chile first round & Michelle Bachelet & True& True\\
		2013 Chile second round & Michelle Bachelet  & True & True\\
		2013 Ecuador & Rafael Correa & True& True\\
		2013 Honduras & Juan Orlando Hernández  & False& \\
		2013 Paraguay & Horacio Cartes & True& True\\
		2014 Colombia first round & Óscar Iván Zuluaga & False& True\\
		2014 Colombia second round & Juan Manuel Santos & False& False\\
		2014 Panama & Juan Carlos Varela& False& True\\\hline
		Average Accuracy &    & \%& \%\\
		\hline
	\end{tabular}
	\caption{Track Record of Prediction Algorithms(Need to complete experiments on the missing countries)}
	\label{table:trackRecord2}
\end{table*}

\end{comment}
		
\begin{table*}[ht]
		\centering
		\begin{tabular}{| l | l | r | r | r | r | r | r |}
		\hline
		Election & Candidate & Actual Result & Seed Vocab. & Error & PSL Vocab. & Error \\
		\hline
		\multirow{2}{*}{Mexico\_Jul01} & Pe\~{n}a Nieto & 38.1 & 46.80 & 8.65 & 39.00 & \textbf{0.85} \\\cline{2-7}
											   & L\'{o}pez Obrador & 31.64 & 24.67 & 6.97 & 28.64 & \textbf{3.00} \\
		\hline
		\multirow{2}{*}{Venezuela\_Oct7} & Hugo Ch\'{a}vez & 55.07 & 49.89 & 5.18 & 55.89 & \textbf{0.82}\\\cline{2-7}
																& Henrique Capriles & 44.31 & 36.31 & 8.00 & 43.91 & \textbf{0.40} \\
		\hline
		\multirow{2}{*}{Ecuador\_Feb17} & Rafael Correa & 57.16 & 53.33 & 3.84 & 54.33 & \textbf{2.84} \\\cline{2-7}
												 & Guillermo Lasso & 22.68 & 12.27 & 10.41 & 12.75 & \textbf{9.93} \\
		\hline
		 \multirow{2}{*}{Venezuela\_Apr15} & Nicol\'{a}s Maduro & 50.61 & 51.45 & 0.84 & 50.58 & \textbf{0.03} \\\cline{2-7}
																	& Henrique Capriles & 49.12 & 35.96 & 13.16 & 38.11 & \textbf{11.01} \\
		\hline
		\multirow{2}{*}{Paraguay\_Apr21} & Horacio Cartes & 48.48 & 35.21 & 13.27 & 40.63 & \textbf{7.85} \\\cline{2-7}
												   & Efra\'{i}n Alegre & 39.05 & 31.33 & 7.72 & 34.44 & \textbf{4.62} \\
		\hline
		\multirow{2}{*}{Chile\_Nov17} & Michelle Bachelet & 46.70 & 38.91 & 7.79 & 41.80 & \textbf{4.91}\\\cline{2-7}
														  & Evelyn Matthei & 25.03 & 19.20 & 5.83 & 20.98 & \textbf{4.05} \\
		\hline 
		\multirow{2}{*}{Honduras\_Nov24} & Orlando Hern\'{a}ndez & 36.80 & 25.16 & 11.64 & 28.30 & \textbf{8.50} \\\cline{2-7}
												   & Xiomara Castro & 28.70 & 16.53 & 12.17 & 24.90 & \textbf{3.80} \\
		\hline
		\multirow{2}{*}{Chile\_Dec15} & Michelle Bachelet & 62.16 & 39.12 & 23.04 & 39.80 & \textbf{22.37}\\\cline{2-7}
															& Evelyn Matthei & 37.83 & 20.88 & 16.95 & 21.68 & \textbf{16.15} \\
		\hline
		\multirow{2}{*}{Panama\_May04} & Juan Carlos Varela & 39.07 &31.28  & 7.79 & 36.23  & \textbf{2.84}\\\cline{2-7}
																	& Jose Domingo Arias  & 31.40 & 35.02  & 3.62  & 33.67  & \textbf{2.27} \\
		\hline 											 
		\multirow{2}{*}{Colombia\_Jun15} & Juan Manuel Santos & 50.95 &48.85  & \textbf{2.1} & 45.75 & 5.2\\\cline{2-7}
																	& Oscar Ivan Zuluaga & 45.00 & 43.79 & \textbf{1.21} & 46.72  & 1.72 \\
		\hline
		\end{tabular}
		\caption{Reduction in prediction error by regressing Tweet
features derived from the PSL vocabulary to opinion polls. All values shown are percentages. Observe that in all but the Colombian election, the PSL vocabulary provides
a better estimate of the vote share of the candidate.}
		\label{table:trackRecord}
	\end{table*}	

We adapt one basic forecasting algorithm from current literature to evaluate the
performance of our expanded vocabularies. We emphasize that our focus in this paper is
not on election forecasting per se but an interesting byproduct of our algorithm is that
it provides useful features for election forecasting.
To evaluate this aspect, we develop
a regression approach
to
learn a mapping
from tweet features (more below) to opinion polls, and thus forecast elections. 
this approach is adapted from \cite{bermingham2011using} and \cite{o2010tweets}.

We reason that by regressing from Twitter-derived features to the opinion polls the bias due to Twitter being a non-representative sample
can be mitigated.
We use a total of six features: Klout scores, number of unique users, total number of mentions, sentiment, and incumbency.
The Klout scores, unique users, and mentions are further categorized into positive and negative mentions.
Using our expanded vocabularies to suitably apportion tweets across the candidates,
we normalize each of these features to obtain the relative share of the volume. 
When there is more than one polling house publishing an opinion poll (for the same date) we take the average of the polls. 

\noindent
{\bf Performance:}
The model was tested exhaustively on a total of ten presidential elections from Latin America during 2012 and 2014. We run the prediction model in two different setting: one using the 
seed words and the other using PSL query expansion.
As outlined in the methodology earlier, only tweets geocoded to the specific location were
used to developed predictions for that location.
Table~\ref{table:trackRecord} shows the overall performance of the regression model under two settings. For each election, we predict the percentage of votes
gathered by each candidate. The error rate is calculated as the difference between the predicted
percentage and the actual polled percentage.
In all cases except the Colombia election, the PSL methodology described here gave a better
estimate of the vote share, in many cases within a margin of 5\%.
