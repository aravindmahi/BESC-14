%!TEX root = ../election_besc14.tex

\section{Introduction}
It is now established that social media such as Twitter serves as a weak
predictor or as a correlative surrogate for many real-world trends such
as %book sales~\cite{gruhl2005predictive}, 
box office earnings~\cite{asur2010predicting}, flu case counts~\cite{culotta2010towards}, and even stock 
prices~\cite{bollen2011twitter}. A topic of current interest is the study of how
online chatter can be used to model
the social, economic or political landscape of a country.

Most approaches for tracking real-world phenomena over Twitter rely on first defining a vocabulary
of keywords (or hashtags) to track over the social media. This approach is typically sufficient for
phenomena about which we have a fairly stable understanding. But for rapidly evolving phenomena, this approach alone is insufficient,
e.g., an election season where new developments can disrupt or bolster the preferences for competing candidates
among key groups or populations.
In such cases, rather than %use 
using a static vocabulary to {\it a priori} define social groups
of interest (e.g., Twitter users who are favorably or not favorably disposed
toward specific candidates), it is preferable to grow dynamic vocabularies by modeling the temporal progression of
events. In addition to better defining social group memberships, such modeling can provide
higher-quality data for downstream applications such as forecasting.

We propose a dynamic query expansion strategy that aids in modeling social groups of interest over time, and
we use the domain of elections to demonstrate the effectiveness of our approach. Modeling elections provides both
qualitative and quantitative insight into the utility of our approach.
Our key contributions are as follows:
\begin{enumerate}
\item We demonstrate
a novel unsupervised learning algorithm that builds dynamic vocabularies using probabilistic soft logic (PSL)~\cite{kimmig2012short}, a framework for probabilistic reasoning over relational domains. Beginning with a small seed set of keywords/hashtags, we demonstrate how our PSL program helps grow the seed set into  vocabularies involving hundreds of relevant terms. This aids in significantly improving the retrieval of tweets corresponding to relevant
social groups of interest.
\item In contrast to traditional co-occurrence based query expansion strategies, we develop an approach that harnesses the social structure implicit in group memberships as captured through retweets, and tweet sentiment.
\item Using ten presidential elections from eight countries of Latin America (Mexico, Venezuela, Ecuador, Paraguay, Chile, Panama, Colombia and Honduras), we show how our query expansion methodology helps capture dynamic trends and improve election forecasting performance. 
\end{enumerate}

The rest of this paper is organized as follows. Section 2 covers the related work in query expansion, forecasting elections using social media and social group modeling. Section 3 overviews the probabilistic soft logic (PSL) framework. Section 4 provides details on how we use PSL for election vocabulary expansion. In Section 5, we present our experimental findings, using our approach on various elections. Section 6 concludes with a brief discussion.

\section{Related Work}
Related work can be organized into following key categories: query expansion, forecasting elections using social media,
and social group modeling.
\begin{comment}
\reviews{refer\\
Dongsheng Duan, Yuhua Li, Ruixuan Li, Rui Zhang, and Aiming Wen. 2012. RankTopic: Ranking Based Topic Modeling. In Proceedings of the 2012 IEEE 12th International Conference on Data Mining (ICDM 12) [This work also captures relational information] ; Daniel Ramage, Susan T. Dumais, and Daniel J. Liebling. ICWSM, The AAAI Press, (2010); Castella, Quim and Sutton, Charles A Word Storms: Multiples of Word Clouds for Visual Comparison of Documents. CoRR abs/1301.0503 (2013)\\
Lau, Jey Han, Nigel Collier and Timothy Baldwin (2012) On-line Trend Analysis with Topic Models: hashtag twitter trends detection topic model online, In Proceedings of the 24th International Conference on Computational Linguistics (COLING 2012)
}
\end{comment}

\subsection{Query Expansion}
Query expansion is a classical technique in information retrieval (IR) ~\cite{manning2008introduction} for improving
retrieval performance by overcoming problems such as {\it synonymy}. Query expansion algorithms
are typically iterative in nature wherein a seed set of query terms help identify an initial set
of documents matching the query, and the highly-ranked retrieved documents (either judged by a human
or by `pseudo-relevance' techniques) are used to automatically grow the vocabulary. Modern
query expansion algorithms use sophisticated concept modeling approaches ~\cite{metzler2007latent}
to grow the given seed set. Massoudi et al.~\cite{massoudi2011incorporating} consider not only the co-occurrence but also the time information to score the related terms to expand the query.  
In contrast to such classical approaches,
our proposed approach is based on a dynamic query expansion strategy  intended to track a vocabulary
over time. Furthermore, unlike most approaches to query expansion, our PSL approach uses a probabilistic formalism
to grow the vocabulary. Finally, PSL provides a rich programmatic environment to incorporate 
multiple indicators (social network, demographics, time) to grow the vocabulary rather than pre-committing to a 
specific strategy.

\subsection{Forecasting Elections using Social Media}
Traditional approaches to forecasting elections use ``volume-based'' ideas
~\cite{tumasjan2010predicting,o2010tweets,saez2011total,bermingham2011using}, 
i.e., forecasting election results by assessing the popularities of candidates and their policies.
Some approaches fit a regression model to opinion 
polls with volume of mentions and sentiment as independent variables and the opinion 
polls as the dependent variable~\cite{o2010tweets,bermingham2011using}. 
More sophisticated approaches ~\cite{livne2011party,conover2011predicting,diaz2012taking}
either model the candidates or the voters in the elections rather than compute the aggregated sentiment of the mass.  
Conover et al.~\cite{conover2011predicting} build a support vector machine (SVM)
classifier trained on manually labeled tweets and classify users into ``left'' and ``right'' aligned.
Using this information and how political information diffuses in a network, they demonstrate an accuracy of 95\%  in 
predicting the political alignment of Twitter users.
Livne et al.~\cite{livne2011party} analyze the Twitter profiles of candidates who contested 
the 2010 mid-term elections in the U.S. 
They identify topics specific to groups of candidates, split according to their known political orientations and use 
these features obtained as inputs to a regression model to forecast the elections. 
In a similar technique, Diaz-Aviles et al.~\cite{diaz2012taking} model the candidates by building an emotional vector 
for each candidate using the mentions of that candidate and sentiments associated with each mention learned using 
the NRC Emotion Lexicon (EmoLex). 
They then use %such profiles learned 
these profiles to predict the rise and fall of a candidate's popularity.
In another thread of research, Mustafaraj et al.~\cite{mustafaraj2011vocal} model the distribution of political content 
among Twitter users. 
They divide the users into two groups, the ``vocal minority" and the ``silent majority" and
observe that these two groups engage in different ways over social media.
The vocal minority aims to broaden the impact of tweets by re-tweeting and linking to other web content, whereas 
the silent majority who tweet significantly less are more inclined to share their personal view points.

\begin{comment}
\subsection{Flawed Studies?}
Recent coverage of election forecasting using Twitter has been critical, e.g.,
%see~\cite{metaxas2011not,gayo2012wanted,gayo2011don,gayo2011limits}.
see~\cite{metaxas2011not,gayo2012wanted}
These publications not only list pertinent problems
in using Twitter to forecast elections but also detail recommendations on how to make such methodologies better.
Gayo-Avello surveys %almost all 
the state-of-the-art approaches in predicting elections, 
%in his paper~\cite{gayo2012wanted}, 
most of which have been detailed above.
Gayo-Avello argues that post-hoc analysis of elections in retrospect must not count as valid predictions and that researchers 
must be wary of the \emph{file drawer} effect, i.e., the act of filing away negative results and publishing 
only the positive results. His major points of contention against such models are:
lack of explainability, no direct way to model a `vote' in social media, self-selection bias,
unrepresentativeness of Twitter demographics, lack of sophisticated sentiment modeling strategies
(e.g., to detect humor and sarcasm that abound in political conversations), and inability to capture
indifferences among the voting public (i.e., 
abstaining from tweeting about politics can carry as much signal as explicit mentions
of candidates). Finally, it has been shown that many of these models %sometimes 
do not outperform a simple
base model that forecasts success for the incumbent.
Our approach here is to aid in better modeling of social groups and improve such predictions.
%and regress toward opinion polls using this information.
A truly comprehensive system will utilize social media as just one of its strategies in forecasting and
we do not make any claims of developing a universal forecasting system for elections.
\end{comment}

\subsection{Social Group Modeling}
Twitter is a natural venue to study social group modeling and there have been many studies that aim to
study social groups in the context of election seasons.
Huang et al.~\cite{huang2012social}, for instance,
model user affiliations within groups by capturing social networks comprising users, their posts,
messages to other users and the various groups they intend to model.
Dynamics of group affiliations are captured through various interactions between these entities in the underlying
network. The authors discover hashtag usage specific to political seasons.
Our work here differs in that we model the dynamic growth of vocabularies as events unfold.

